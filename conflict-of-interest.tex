\documentclass{article}
\usepackage{titlesec,soulutf8}
\renewcommand{\thesection}{\Roman{section} }
\renewcommand{\thesubsection}{\arabic{subsection}}
\titleformat{\section}
{\normalfont\bfseries\filcenter}{ARTICLE \thesection:}{1em}{}
\titleformat{\subsection}
{\normalfont}{\ul{Section \thesubsection. \enspace}}{0em}{\ul}
\begin{document}
\title{CONFLICT OF INTEREST POLICY \protect\\ OF \protect\\ REGEN FOUNDATION}
\author{ Adopted: February 13th, 2020}
\date{} 
\maketitle
\clearpage
\section{PURPOSE}
The purpose of the conflicts of interest policy (the “Policy”) is to protect the interests of Regen Foundation (the “Corporation”) when it is contemplating entering into a transaction or arrangement that might benefit the private interest of an officer or director of the Corporation.  This Policy is intended to supplement but not replace any applicable state laws governing conflicts of interest applicable to nonprofit and charitable corporations.
\section{DEFINITIONS}
\subsection{Interested Person}
Any director, principal officer, or member of a committee with board delegated powers who has a direct or indirect financial interest, as defined below, is an interested person.
\subsection{Financial Interest}
A person has a financial interest if the person has, directly or indirectly, through business, investment or family:
An ownership or investment interest in any entity with which the Corporation has a transaction or arrangement;
A compensation arrangement with the Corporation or with any entity or individual with which the Corporation has a transaction or arrangement; or
A potential ownership or investment interest in, or compensation arrangement with, any entity or individual with which the Corporation is negotiating a transaction or arrangement.
Compensation includes direct and indirect remuneration as well as gifts or favors that are substantial in nature.
A financial interest is not necessarily a conflict of interest.  Under Article III, Section 2, a person who has a financial interest may have a conflict of interest only if the appropriate board or committee decides that a conflict of interest exists.
\section{PROCEDURES}
\subsection{Duty to Disclose}
In connection with any actual or possible conflicts of interest, an interested person must disclose the existence of his or her financial interest and must be given the opportunity to disclose all material facts to the directors and members of committees with board delegated powers considering the proposed transaction or arrangement.  
\subsection{Determining Whether a Conflict of Interest Exists}
After disclosure of the financial interest and all material facts, and after any discussion with the interested person, he/she shall leave the board or committee meeting while the determination of a conflict of interest is discussed and voted upon. The remaining board or committee members shall decide if a conflict of interest exists.
\subsection{Procedures for Addressing the Conflict of Interest}
An interested person may make a presentation at the board or committee meeting, but after such presentation, he/she shall leave the meeting during the discussion of, and the vote on, the transaction or arrangement that results in the conflict of interest.
The chairperson of the board or committee shall, if appropriate, appoint a disinterested person or committee to investigate alternatives to the proposed transaction or arrangement.
After exercising due diligence, the board or committee shall determine whether the Corporation can obtain a more advantageous transaction or arrangement with reasonable efforts from a person or entity that would not give rise to a conflict of interest.
If a more advantageous transaction or arrangement is not reasonably attainable under circumstances that would not give rise to a conflict of interest, the board or committee shall determine by a majority vote of the disinterested directors whether the transaction or arrangement is in the Corporation’s best interest and for its own benefit and whether the transaction is fair and reasonable to the Corporation and shall make its decision as to whether to enter into the transaction or arrangement in conformity with such determination.
In all events, the Corporation will attempt to comply with the requirements for invoking the rebuttable presumption under section 53.4958-6 of the Treasury regulations.
\subsection{Violations of the Conflict of Interest Policy}
If the board or committee has reasonable cause to believe that a member has failed to disclose actual or possible conflicts of interest, it shall inform the member of the basis for such belief and afford the member an opportunity to explain the alleged failure to disclose.
If, after hearing the response of the member and making such further investigation as may be warranted in the circumstances, the board or committee determines that the member has in fact failed to disclose an actual or possible conflict of interest it shall take appropriate disciplinary and corrective action.
\section{RECORDS OF PROCEEDINGS}
The minutes of the board and all committees with board-delegated powers shall contain:
\renewcommand{\labelenumi}{\alph{enumi}$)$}
\begin{enumerate}
\item The names of the persons who disclosed or otherwise were found to have a financial interest in connection with an actual or possible conflict of interest, the nature of the financial interest, any action taken to determine whether a conflict of interest was present, and the board’s or committee’s decision as to whether a conflict of interest in fact existed; and
\item The names of the persons who were present for discussions and votes relating to the transaction or arrangement, the content of the discussion, including any alternatives to the proposed transaction or arrangement, and a record of any votes taken in connection therewith.
\end{enumerate}
\section{COMPENSATION}
Any voting member of the board of directors who receives compensation, directly or indirectly, from the Corporation for services is precluded from discussing and voting on matters pertaining to that member’s compensation.
A voting member of any committee whose jurisdiction includes compensation matters and who receives compensation, directly or indirectly, from the Corporation for services is precluded from voting on matters pertaining to that member’s compensation.
\section{ANNUAL STATEMENTS}
Each director, principal officer and member of a committee with board delegated powers shall affirm that such person:
\renewcommand{\labelenumi}{\alph{enumi}$)$}
\begin{enumerate}
\item Has received a copy of the conflicts of interest policy;
\item Has read and understands the policy;
\item Has agreed to comply with the policy; and
\item Understands that the Corporation is a charitable organization and that in order to maintain its federal tax exemption it must engage primarily in activities which accomplish one or more of its tax-exempt purposes.
\end{enumerate}
\section{PERIODIC REVIEWS}
To ensure that the Corporation operates in a manner consistent with its charitable purposes and that it does not engage in activities that could jeopardize its status as an organization exempt from federal income tax, periodic reviews shall be conducted.  The periodic reviews shall, at a minimum, include the following subjects:
\renewcommand{\labelenumi}{\alph{enumi}$)$}
\begin{enumerate}
\item Whether compensation arrangements and benefits are reasonable and are the result of arm’s-length bargaining
\item Whether acquisitions result in inurement or impermissible private benefit;
\item Whether all transactions are properly recorded, reflect reasonable payments for goods and services, further the Corporation’s charitable purposes and do not result in inurement or impermissible private benefit; and
\item Whether agreements entered into by the Corporation furthers the Corporation’s charitable purposes and do not result in inurement or impermissible private benefit.
\end{enumerate}
\section{USE OF OUTSIDE EXPERTS}
In conducting the periodic reviews provided for in Article VII, the Corporation may, but need not, use outside advisors.  If outside experts are used their use shall not relieve the board of its responsibility for ensuring that periodic reviews are conducted.
\section{ADOPTION OF POLICY}
This Policy has been adopted on the 13th day of February 2020 by the unanimous written consent of the Board of Directors and a copy of this Policy will be distributed to each director and officer of the Corporation.
\end{document}